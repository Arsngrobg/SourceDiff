\documentclass[progress]{cmpreport}

\date             {2025-10-03}
\title            {Analysing Source Code for AI detection, Plagiarism, Collusion, and Refactoring Using Parse Trees}
\author           {James Jack Armstrong}
\supervisor       {Prof. Gavin Cawley}
\registration     {100425906}
\ccode            {CMP6001Y}
\summary          {Being able to perform advanced code analysis for measuring and minimizing distance between
                   codebases. Through leveraging static analysis, parse trees (PTs), and pattern recognition
                   techniques, being able to provide metrics and helpful refactoring suggestions to align codebases
                   more closely or to highlight significant divergences.
                  }
\acknowledgements {I would like to thank my supervisor Prof. Gavin Cawley, a professor from the University of East
                   Anglia (UEA), for overseeing the development of this project and research.
                  }

\begin{document}
    \bibliography{report}

    \newpage
    \section{Tree Sitter API}\label{sec:tree-sitter-api} {
        Tree Sitter is a parsing generator tool and an incremental parsing library. It is advertised as fast & efficient
        at building and updating concrete syntax trees as modifications are made to the source. This API is perfect for
        this type of software as it allows for a wide range of support for various languages without relying on 1st
        party/in-house support for other languages. For now, I will be using the C parser for demonstration purposes.
    }
\end{document}
