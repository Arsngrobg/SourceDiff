\documentclass[final]{class/cmpreport}

\date             {2025-10-03}
\title            {Analysing Source Code for AI detection, Plagiarism, Collusion, and Refactoring Using Parse Trees}
\author           {James Jack Armstrong}
\supervisor       {Prof. Gavin Cawley}
\registration     {100425906}
\ccode            {CMP6001Y}
\summary          {Being able to perform advanced code analysis for measuring and minimizing distance between
                   codebases. Through leveraging static analysis, parse trees (PTs), and pattern recognition
                   techniques, being able to provide metrics and helpful refactoring suggestions to align codebases
                   more closely or to highlight significant divergences.
                  }
\acknowledgements {I would like to thank my supervisor Prof. Gavin Cawley, a professor from the University of East
                   Anglia (UEA), for overseeing the development of this project and research.
                  }

\begin{document}
    \section{Introduction}\label{sec:introduction}
        SourceDiff is a tool for performing advanced code analysis for measuring and minimizing the distance between
        codebases.
        It leverages static analysis of source code using parse trees, and pattern recognition techniques to provide
        useful metrics and helpful refactoring suggestions to align codebases more closely, or highlight significant
        divergences in code structure.
        \newline\newline
        The primary aim of this project is to analyse source code, primarily for plagiarism detection, either by
        copy-pasting from programming forums such as StackOverflow\citep{StackOverflow-CopyPaste},
        or Reddit\citep{Reddit-CopyPaste}.
        Alternatively, through the use of generative AI tools such as \("\)ChatGPT and other artificial intelligence
        tools \ldots while traditional forms of plagiarism show a marked decline.\("\)\citep{Students-Cheating-Using-ChatGPT}
        By using parse trees we can analyse the source code based on structure and syntax.
        \newline\newline
        Required areas of knowledge for this report are: concrete syntax tress (CTS - or Parse Trees), and how they
        differ from abstract syntax trees; and various tree/graph analysis algorithms.

    \bibliography{doc}
\end{document}
